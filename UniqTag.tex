\documentclass[10pt]{article}

\usepackage{graphicx} % for \includegraphics

% amsmath package, useful for mathematical formulas
\usepackage{amsmath}
% amssymb package, useful for mathematical symbols
\usepackage{amssymb}

% cite package, to clean up citations in the main text. Do not remove.
\usepackage{cite}

\usepackage{hyperref}

% line numbers
\usepackage{lineno}

% ligatures disabled
\usepackage{microtype}
\DisableLigatures[f]{encoding = *, family = * }

% Text layout
\topmargin 0.0cm
\oddsidemargin 0.5cm
\evensidemargin 0.5cm
\textwidth 16cm
\textheight 21cm

% Bold the 'Figure #' in the caption and separate it with a period
% Captions will be left justified
\usepackage[labelfont=bf,labelsep=period,justification=raggedright]{caption}

% Use the PLoS provided BiBTeX style
\bibliographystyle{plos2009}

% Remove brackets from numbering in List of References
\makeatletter
\renewcommand{\@biblabel}[1]{\quad#1.}
\makeatother

% Leave date blank
\date{}

\pagestyle{myheadings}

\setcounter{secnumdepth}{0}

\begin{document}

\begin{flushleft}
{\large
\textbf{UniqTag: Content-derived unique and stable identifiers for gene annotation}
}
\\
Shaun D. Jackman$^{1,2,\ast}$,
Joerg Bohlmann$^{3}$,
\.{I}nan\c{c} Birol$^{1,4,\dagger}$
\\ \textbf{1} Genome Sciences Centre, British Columbia Cancer Agency, Vancouver, BC, Canada
\\ \textbf{2} Graduate Program in Bioinformatics, University of British Columbia, Vancouver, BC, Canada
\\ \textbf{3} Michael Smith Laboratories, University of British Columbia, Vancouver, BC, Canada
\\ \textbf{4} Department of Medical Genetics, University of British Columbia, Vancouver, BC, Canada
\\ $\ast$ E-mail: sjackman@bcgsc.ca, Twitter: @sjackman
\\ $\dagger$ E-mail: ibirol@bcgsc.ca
\end{flushleft}
\section{Abstract}\label{abstract}

When working on an ongoing genome sequencing and assembly project, it is
rather inconvenient when gene identifiers change from one build of the
assembly to the next. The gene labelling system described here, UniqTag,
addresses this common challenge. UniqTag assigns a unique identifier to
each gene that is a representative \emph{k}-mer, a string of length
\emph{k}, selected from the sequence of that gene. Unlike serial
numbers, these identifiers are stable between different assemblies and
annotations of the same data without requiring that previous annotations
be lifted over by sequence alignment. We assign UniqTag identifiers to
ten builds of the Ensembl human genome spanning eight years to
demonstrate this stability.

The implementation of UniqTag in Ruby and an R package are available at
\url{https://github.com/sjackman/uniqtag}. The R package is also
available from CRAN: \texttt{install.packages("uniqtag")}. Supplementary
material and code to reproduce it is available at
\url{https://github.com/sjackman/uniqtag-paper}.

\section{Introduction}\label{introduction}

The task of annotating the genes of a genome sequence often follows
genome assembly. These annotated genes are assigned unique identifiers
by which they can be referenced. Assembly and annotation is frequently
an iterative process, by refining the method or by the addition of more
sequencing data. These gene identifiers would ideally be stable from one
assembly and annotation to the next. The common practice is to use
serial numbers to identify genes that are annotated by software such as
MAKER {[}\href{http://dx.doi.org/10.1104/pp.113.230144}{1}{]}, which,
although certainly unique, are not stable between assemblies. A single
change in the assembly can result in a total renumbering of the
annotated genes.

One solution to stabilize identifiers is to assign them based on the
content of the gene sequence. A cryptographic hash function such as SHA
(Secure Hash Algorithm)
{[}\href{http://www.nist.gov/manuscript-publication-search.cfm?pub_id=910977}{2}{]}
derives a message digest from the sequence, such that two sequences with
the same content will have the same message digest, and two sequences
that differ will have different message digests. If a cryptographic hash
were used to identify a gene, the same gene in two assemblies with
identical content would be assigned identical identifiers. Yet, by
design a slight change in the sequence, such as a single-character
substitution, would result in a completely different digest.

Locality-sensitive hashing in contrast aims to assign items that are
similar to the same hash value. A hash function that assigns an
identical identifier to a sequence after a modification of that sequence
is desirable for labelling the genes of an ongoing genome annotation
project. One such locality-sensitive hash function, MinHash, was
employed to identify web pages with similar content
{[}\href{http://dx.doi.org/10.1109/SEQUEN.1997.666900}{3}{]} by
selecting a small representative set of words from a web page.

UniqTag is inspired by MinHash. It selects a single representative
\emph{k}-mer from a sequence to assign a stable identifier to a gene.
These identifiers are intended for systematic labelling of genes rather
than assigning biological gene names, as the latter are typically based
on biological function or homology to orthologous genes
{[}\href{http://dx.doi.org/10.1006/geno.2002.6748}{4}{]}. Assigning
UniqTag identifiers to the current assembly requires no knowledge of the
previous assemblies.

Annotation of a draft genome that is in progress requires a system by
which to identify genes, even if only temporarily and for internal use.
Assigning permanent identifiers to a stable genome assembly for public
release, such as with complete genome assemblies of model organisms, is
a different situation and typically involves a public database of
permanently accessioned identifiers, lifting over gene identifiers from
one assembly to an updated assembly using sequence alignment, rules of
versioning those symbols, and often manual curation. UniqTag on the
other hand provides a simple system for quickly assigning stable
identifiers with little more effort than assigning serial numbers. Our
approach does not require sequence alignment nor in fact any knowledge
of previous assemblies to assign identifiers to new assemblies.

\section{Methods}\label{methods}

The UniqTag is defined mathematically as follows. Let \(\Sigma\) be an
alphabet, such as the twenty standard amino acids or the four
nucleotides. Let \(\Sigma^k\) be the set of all strings over \(\Sigma\)
of length \emph{k}. Let \emph{s} be a string over \(\Sigma\), such as
the peptide or nucleotide sequence of a gene. Let \(C(s)\) be the set of
all substrings of \emph{s}, and \(C_k(s)\) be the set of all
\emph{k}-mers of \emph{s}, that is, all substrings of \emph{s} with
length \emph{k}.

\[
C_k(s) = C(s) \cap \Sigma^k
\]

Let \emph{S} be a set of \emph{n} strings \(\{s_0, \dots, s_n\}\), such
as the peptide or nucleotide sequences of all the annotated genes of a
genome assembly. Let \(f(t, S)\) be the frequency in \emph{S} of a
\emph{k}-mer \emph{t}, defined as the number of strings in \emph{S} that
contain the \emph{k}-mer \emph{t}.

\[
f(t, S) = \left\vert \{ s \mid t \in C_k(s) \wedge s \in S \} \right\vert
\]

Let the function \(\min T\) define the lexicographically minimal string
of a set of strings \emph{T}. That is, if the strings of the set
\emph{T} were sorted alphabetically, \(\min T\) would refer to the first
string in the list.

Finally, \(u_k(s, S)\) is the UniqTag, the lexicographically minimal
\emph{k}-mer of those \emph{k}-mers of \emph{s} that are least frequent
in \emph{S}.

\[
u_k(s, S) = \min \mathop{\arg\,\min}\limits_{t \in C_k(s)} f(t, S)
\]

Typically, \(u_k(s, S)\) is the first \emph{k}-mer in an alphabetically
sorted list of the \emph{k}-mers of a gene that are unique to that gene.

A UniqTag can be generated from the nucleotide sequence of a gene or the
translated peptide sequence of a protein-coding gene. Using the peptide
sequence results in a UniqTag that is stable across synonymous changes
to the coding sequence as well as to changes in the untranslated regions
and introns of the gene. Since the amino acid alphabet is larger than
the nucleotide alphabet, fewer characters are required for a
\emph{k}-mer to be likely unique, resulting in an aesthetically pleasing
shorter identifier.

When two gene models have identical \emph{k}-mer compositions, they
would be assigned the same UniqTag. It is also possible that two genes
that have no unique \emph{k}-mer and similar \emph{k}-mer composition
are assigned the same UniqTag. In such cases, genes that have the same
UniqTag are distinguished by adding a numerical suffix to the UniqTag. A
UniqTag is formatted as a \emph{k}-mer followed by a hyphen and a
number, such as \emph{ARNDCEQGH-1}. For consistency of formatting the
suffix is always included, even when the \emph{k}-mer is unique and the
numerical suffix is \emph{-1}.

Some genes, such as those found in transposable elements, may occur
hundreds of times with little variation. When the repetitive elements
are not perfectly identical, the \emph{k}-mers that are unique to each
instance of the repetitive element will be chosen preferentially as
their UniqTags. Genome sequence assembly may smooth out small variants
by collapsing repeats during the initial assembly and then expanding
repeats using paired-end reads, after which there may be many identical
copies of the repetitive element with no distinguishing unique
\emph{k}-mer. Identifying and masking repetitive elements prior to gene
annotation will avoid assigning gene identifiers to genes found in
repeat elements. Alternatively if assigning gene identifiers to genes
found in repeat elements is desirable, a unique \emph{k}-mer could be
selected from flanking DNA sequence when the coding sequence is found to
be too repetitive. Although this feature is not integrated into our
software, this approach can be implemented by assigning UniqTags to both
the translated coding sequence and the flanking DNA sequence and
selecting the DNA UniqTag when the coding UniqTag is found to be
repetitive.

The UniqTag is designed to be stable but will change in the following
conditions: (1)~when the sequence at the locus of the UniqTag changes;
(2)~when a least-frequent \emph{k}-mer that is lexicographically smaller
than the previous UniqTag is created; (3)~when a duplicate \emph{k}-mer
is created elsewhere that results in the previous UniqTag no longer
being a least-frequent \emph{k}-mer.

UniqTag identifiers are stable to small changes to the gene sequence,
such as the correction of small misassemblies, but large changes to the
gene sequence or gene model, such as the addition or removal of entire
exons, will often result in a change of the UniqTag identifier of that
gene. The larger the change of the sequence or gene model, the more
likely it is that the UniqTag will change.

The special cases of merging and splitting gene models are interesting.
Concatenating two gene models results in a gene whose UniqTag is the
minimum of the two previous UniqTags, unless the new UniqTag spans the
junction of the two sequences. Similarly when a gene model is split in
two, one gene is assigned a new UniqTag and the other retains the
previous UniqTag, unless the previous UniqTag spanned the junction.

Importantly and in contrast, unlike naming the genes after the genomic
contigs or scaffolds in which they are found, changing the names or the
order of the sequences in a genome assembly has no effect on the UniqTag
\emph{k}-mers of those genes.

\section{Results}\label{results}

To demonstrate the stability and utility of UniqTag, we assigned
identifiers to the genes of ten builds of the Ensembl human genome
{[}\href{http://dx.doi.org/10.1093/nar/gku1010}{5}{]} (every fifth build
from 40 through 70, and builds 74, 75 and 76) spanning eight years and
three major genome assemblies (NCBI36 up to build 54, GRCh37 up to build
75, and GRCh38 for build 76). Ensembl build 75, the final build to use
GRCh37, is used as the reference to which all other builds are compared.
The number of common UniqTag identifiers between build 75 and nine other
builds is shown in Fig.~1. A UniqTag identifier of nine amino acids
(\(k=9\)) was assigned to the first protein sequence, that with the
smallest Ensembl protein (ENSP) accession number, of each gene. Also
shown is the number of common gene and protein identifiers (ENSG and
ENSP accession numbers) between builds and the number of genes with
peptide sequences that are identical between builds. Although less
stable than the gene ID, the UniqTag is more stable than the protein ID
and the peptide sequence. The last build of GRCh37, build 75, and the
first build of GRCh38, build 76, for example have 20,376 (90.7\%)
UniqTag in common and 21,097 (93.9\%) ENSG accession numbers in common
of the 22,469 genes of build 76.

\begin{figure}[htbp]
\centering
\includegraphics{figure/ensembl.png}
\caption{The number of common UniqTag identifiers between build 75 of
the Ensembl human genome and nine other builds, the number of common
gene and protein identifiers between builds, and the number of genes
with peptide sequences that are identical between builds.}
\end{figure}

The stability of the UniqTag is insensitive to the size of the UniqTag
identifier for values of \emph{k} between 8 and 50 amino acids, shown in
Fig.~2. The data for both figures are shown in Table~A in S1~File.

\begin{figure}[htbp]
\centering
\includegraphics{figure/k.png}
\caption{The number of common UniqTag identifiers between build 75 of
the Ensembl human genome and nine other builds for different values of
\emph{k}.}
\end{figure}

As described above, genes with the same peptide sequence result in hash
collisions and are disambiguated using a numerical suffix. Duplicate
UniqTag \emph{k}-mers due to hash collisions are rare, but can occur in
sequences that have no unique \emph{k}-mer, which is most likely with
short sequences. NCBI GRCh37 build 75 has 23,393 annotated genes, which
have 21,783 (93.1\%) distinct peptide sequences. Of these 21,783
distinct sequences, there are 54 (0.25\%) UniqTag collisions.

\section{Conclusions}\label{conclusions}

Whereas the gene and protein identifiers can, with effort, be lifted
over from older builds to the newest build, the UniqTag identifier can
be generated without any knowledge of previous assemblies, making it a
much simpler operation. The number of identical peptide sequences
between builds shows the stability that would be expected of using a
cryptographic hash value of the peptide sequence as the identifier. The
stability of the UniqTag is insensitive to the size of the UniqTag
\emph{k}-mer for a large range of \emph{k}.

\section{Acknowledgements}\label{acknowledgements}

The authors thank Nathaniel Street for his enthusiastic feedback, the
SMarTForests project and the organizers of the 2014 Conifer Genome
Summit that made our conversation possible.

\section{References}\label{references}

{[}1{]}~\href{http://dx.doi.org/10.1104/pp.113.230144}{Campbell MS, Law
MY, Holt C, Stein JC, Moghe GD, Hufnagel DE, et al. (2014)} MAKER-P: a
tool-kit for the rapid creation, management, and quality control of
plant genome annotations. Plant Physiology 164: 513-524.
doi:10.1104/pp.113.230144\\{[}2{]}~\href{http://www.nist.gov/manuscript-publication-search.cfm?pub_id=910977}{Dang
QH (2012)} Secure Hash Standard (SHS). NIST FIPS 180:
1-35.\\{[}3{]}~\href{http://dx.doi.org/10.1109/SEQUEN.1997.666900}{Broder
AZ (1997)} On the resemblance and containment of documents. Compression
and Complexity of Sequences 1997 Proceedings: 21-29.
doi:10.1109/SEQUEN.1997.666900\\{[}4{]}~\href{http://dx.doi.org/10.1006/geno.2002.6748}{Wain
HM, Bruford EA, Lovering RC, Lush MJ, Wright MW, Povey S (2002)}
Guidelines for human gene nomenclature. Genomics 79: 464-470.
doi:10.1006/geno.2002.6748\\{[}5{]}~\href{http://dx.doi.org/10.1093/nar/gku1010}{Cunningham
F, Amode MR, Barrell D, Beal K, Billis K, Brent S, et al. (2015)}
Ensembl 2015. Nucleic Acids Research 43: D662-D669.
doi:10.1093/nar/gku1010

\section{Supporting Information
Legends}\label{supporting-information-legends}

\begin{description}
\itemsep1pt\parskip0pt\parsep0pt
\item[S1 File.]
The data used to plot Figs. 1 and 2, also available in tab-separated
values (TSV) format (\textbf{Table~A}). The source of UniqTag 1.0,
implemented in Ruby (\textbf{Listing~A}). The Makefile script that
calculates the data used to plot Figs. 1 and 2 (\textbf{Listing~B}).
\end{description}
\end{document}
